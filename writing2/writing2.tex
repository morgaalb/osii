
%% bare_jrnl.tex
%% V1.4b
%% 2015/08/26
%% by Michael Shell
%% see http://www.michaelshell.org/
%% for current contact information.
%%
%% This is a skeleton file demonstrating the use of IEEEtran.cls
%% (requires IEEEtran.cls version 1.8b or later) with an IEEE
%% journal paper.
%%
%% Support sites:
%% http://www.michaelshell.org/tex/ieeetran/
%% http://www.ctan.org/pkg/ieeetran
%% and
%% http://www.ieee.org/

%%*************************************************************************
%% Legal Notice:
%% This code is offered as-is without any warranty either expressed or
%% implied; without even the implied warranty of MERCHANTABILITY or
%% FITNESS FOR A PARTICULAR PURPOSE! 
%% User assumes all risk.
%% In no event shall the IEEE or any contributor to this code be liable for
%% any damages or losses, including, but not limited to, incidental,
%% consequential, or any other damages, resulting from the use or misuse
%% of any information contained here.
%%
%% All comments are the opinions of their respective authors and are not
%% necessarily endorsed by the IEEE.
%%
%% This work is distributed under the LaTeX Project Public License (LPPL)
%% ( http://www.latex-project.org/ ) version 1.3, and may be freely used,
%% distributed and modified. A copy of the LPPL, version 1.3, is included
%% in the base LaTeX documentation of all distributions of LaTeX released
%% 2003/12/01 or later.
%% Retain all contribution notices and credits.
%% ** Modified files should be clearly indicated as such, including  **
%% ** renaming them and changing author support contact information. **
%%*************************************************************************


% *** Authors should verify (and, if needed, correct) their LaTeX system  ***
% *** with the testflow diagnostic prior to trusting their LaTeX platform ***
% *** with production work. The IEEE's font choices and paper sizes can   ***
% *** trigger bugs that do not appear when using other class files.       ***                          ***
% The testflow support page is at:
% http://www.michaelshell.org/tex/testflow/



\documentclass[10pt,journal,draftclsnofoot,onecolumn]{IEEEtran}
%
% If IEEEtran.cls has not been installed into the LaTeX system files,
% manually specify the path to it like:
% \documentclass[journal]{../sty/IEEEtran}





% Some very useful LaTeX packages include:
% (uncomment the ones you want to load)


% *** MISC UTILITY PACKAGES ***
%
%\usepackage{ifpdf}
% Heiko Oberdiek's ifpdf.sty is very useful if you need conditional
% compilation based on whether the output is pdf or dvi.
% usage:
% \ifpdf
%   % pdf code
% \else
%   % dvi code
% \fi
% The latest version of ifpdf.sty can be obtained from:
% http://www.ctan.org/pkg/ifpdf
% Also, note that IEEEtran.cls V1.7 and later provides a builtin
% \ifCLASSINFOpdf conditional that works the same way.
% When switching from latex to pdflatex and vice-versa, the compiler may
% have to be run twice to clear warning/error messages.






% *** CITATION PACKAGES ***
%
%\usepackage{cite}
% cite.sty was written by Donald Arseneau
% V1.6 and later of IEEEtran pre-defines the format of the cite.sty package
% \cite{} output to follow that of the IEEE. Loading the cite package will
% result in citation numbers being automatically sorted and properly
% "compressed/ranged". e.g., [1], [9], [2], [7], [5], [6] without using
% cite.sty will become [1], [2], [5]--[7], [9] using cite.sty. cite.sty's
% \cite will automatically add leading space, if needed. Use cite.sty's
% noadjust option (cite.sty V3.8 and later) if you want to turn this off
% such as if a citation ever needs to be enclosed in parenthesis.
% cite.sty is already installed on most LaTeX systems. Be sure and use
% version 5.0 (2009-03-20) and later if using hyperref.sty.
% The latest version can be obtained at:
% http://www.ctan.org/pkg/cite
% The documentation is contained in the cite.sty file itself.






% *** GRAPHICS RELATED PACKAGES ***
%
\ifCLASSINFOpdf
  % \usepackage[pdftex]{graphicx}
  % declare the path(s) where your graphic files are
  % \graphicspath{{../pdf/}{../jpeg/}}
  % and their extensions so you won't have to specify these with
  % every instance of \includegraphics
  % \DeclareGraphicsExtensions{.pdf,.jpeg,.png}
\else
  % or other class option (dvipsone, dvipdf, if not using dvips). graphicx
  % will default to the driver specified in the system graphics.cfg if no
  % driver is specified.
  % \usepackage[dvips]{graphicx}
  % declare the path(s) where your graphic files are
  % \graphicspath{{../eps/}}
  % and their extensions so you won't have to specify these with
  % every instance of \includegraphics
  % \DeclareGraphicsExtensions{.eps}
\fi
% graphicx was written by David Carlisle and Sebastian Rahtz. It is
% required if you want graphics, photos, etc. graphicx.sty is already
% installed on most LaTeX systems. The latest version and documentation
% can be obtained at: 
% http://www.ctan.org/pkg/graphicx
% Another good source of documentation is "Using Imported Graphics in
% LaTeX2e" by Keith Reckdahl which can be found at:
% http://www.ctan.org/pkg/epslatex
%
% latex, and pdflatex in dvi mode, support graphics in encapsulated
% postscript (.eps) format. pdflatex in pdf mode supports graphics
% in .pdf, .jpeg, .png and .mps (metapost) formats. Users should ensure
% that all non-photo figures use a vector format (.eps, .pdf, .mps) and
% not a bitmapped formats (.jpeg, .png). The IEEE frowns on bitmapped formats
% which can result in "jaggedy"/blurry rendering of lines and letters as
% well as large increases in file sizes.
%
% You can find documentation about the pdfTeX application at:
% http://www.tug.org/applications/pdftex





% *** MATH PACKAGES ***
%
%\usepackage{amsmath}
% A popular package from the American Mathematical Society that provides
% many useful and powerful commands for dealing with mathematics.
%
% Note that the amsmath package sets \interdisplaylinepenalty to 10000
% thus preventing page breaks from occurring within multiline equations. Use:
%\interdisplaylinepenalty=2500
% after loading amsmath to restore such page breaks as IEEEtran.cls normally
% does. amsmath.sty is already installed on most LaTeX systems. The latest
% version and documentation can be obtained at:
% http://www.ctan.org/pkg/amsmath





% *** SPECIALIZED LIST PACKAGES ***
%
%\usepackage{algorithmic}
% algorithmic.sty was written by Peter Williams and Rogerio Brito.
% This package provides an algorithmic environment fo describing algorithms.
% You can use the algorithmic environment in-text or within a figure
% environment to provide for a floating algorithm. Do NOT use the algorithm
% floating environment provided by algorithm.sty (by the same authors) or
% algorithm2e.sty (by Christophe Fiorio) as the IEEE does not use dedicated
% algorithm float types and packages that provide these will not provide
% correct IEEE style captions. The latest version and documentation of
% algorithmic.sty can be obtained at:
% http://www.ctan.org/pkg/algorithms
% Also of interest may be the (relatively newer and more customizable)
% algorithmicx.sty package by Szasz Janos:
% http://www.ctan.org/pkg/algorithmicx




% *** ALIGNMENT PACKAGES ***
%
%\usepackage{array}
% Frank Mittelbach's and David Carlisle's array.sty patches and improves
% the standard LaTeX2e array and tabular environments to provide better
% appearance and additional user controls. As the default LaTeX2e table
% generation code is lacking to the point of almost being broken with
% respect to the quality of the end results, all users are strongly
% advised to use an enhanced (at the very least that provided by array.sty)
% set of table tools. array.sty is already installed on most systems. The
% latest version and documentation can be obtained at:
% http://www.ctan.org/pkg/array


% IEEEtran contains the IEEEeqnarray family of commands that can be used to
% generate multiline equations as well as matrices, tables, etc., of high
% quality.




% *** SUBFIGURE PACKAGES ***
%\ifCLASSOPTIONcompsoc
%  \usepackage[caption=false,font=normalsize,labelfont=sf,textfont=sf]{subfig}
%\else
%  \usepackage[caption=false,font=footnotesize]{subfig}
%\fi
% subfig.sty, written by Steven Douglas Cochran, is the modern replacement
% for subfigure.sty, the latter of which is no longer maintained and is
% incompatible with some LaTeX packages including fixltx2e. However,
% subfig.sty requires and automatically loads Axel Sommerfeldt's caption.sty
% which will override IEEEtran.cls' handling of captions and this will result
% in non-IEEE style figure/table captions. To prevent this problem, be sure
% and invoke subfig.sty's "caption=false" package option (available since
% subfig.sty version 1.3, 2005/06/28) as this is will preserve IEEEtran.cls
% handling of captions.
% Note that the Computer Society format requires a larger sans serif font
% than the serif footnote size font used in traditional IEEE formatting
% and thus the need to invoke different subfig.sty package options depending
% on whether compsoc mode has been enabled.
%
% The latest version and documentation of subfig.sty can be obtained at:
% http://www.ctan.org/pkg/subfig




% *** FLOAT PACKAGES ***
%
%\usepackage{fixltx2e}
% fixltx2e, the successor to the earlier fix2col.sty, was written by
% Frank Mittelbach and David Carlisle. This package corrects a few problems
% in the LaTeX2e kernel, the most notable of which is that in current
% LaTeX2e releases, the ordering of single and double column floats is not
% guaranteed to be preserved. Thus, an unpatched LaTeX2e can allow a
% single column figure to be placed prior to an earlier double column
% figure.
% Be aware that LaTeX2e kernels dated 2015 and later have fixltx2e.sty's
% corrections already built into the system in which case a warning will
% be issued if an attempt is made to load fixltx2e.sty as it is no longer
% needed.
% The latest version and documentation can be found at:
% http://www.ctan.org/pkg/fixltx2e


%\usepackage{stfloats}
% stfloats.sty was written by Sigitas Tolusis. This package gives LaTeX2e
% the ability to do double column floats at the bottom of the page as well
% as the top. (e.g., "\begin{figure*}[!b]" is not normally possible in
% LaTeX2e). It also provides a command:
%\fnbelowfloat
% to enable the placement of footnotes below bottom floats (the standard
% LaTeX2e kernel puts them above bottom floats). This is an invasive package
% which rewrites many portions of the LaTeX2e float routines. It may not work
% with other packages that modify the LaTeX2e float routines. The latest
% version and documentation can be obtained at:
% http://www.ctan.org/pkg/stfloats
% Do not use the stfloats baselinefloat ability as the IEEE does not allow
% \baselineskip to stretch. Authors submitting work to the IEEE should note
% that the IEEE rarely uses double column equations and that authors should try
% to avoid such use. Do not be tempted to use the cuted.sty or midfloat.sty
% packages (also by Sigitas Tolusis) as the IEEE does not format its papers in
% such ways.
% Do not attempt to use stfloats with fixltx2e as they are incompatible.
% Instead, use Morten Hogholm'a dblfloatfix which combines the features
% of both fixltx2e and stfloats:
%
% \usepackage{dblfloatfix}
% The latest version can be found at:
% http://www.ctan.org/pkg/dblfloatfix




%\ifCLASSOPTIONcaptionsoff
%  \usepackage[nomarkers]{endfloat}
% \let\MYoriglatexcaption\caption
% \renewcommand{\caption}[2][\relax]{\MYoriglatexcaption[#2]{#2}}
%\fi
% endfloat.sty was written by James Darrell McCauley, Jeff Goldberg and 
% Axel Sommerfeldt. This package may be useful when used in conjunction with 
% IEEEtran.cls'  captionsoff option. Some IEEE journals/societies require that
% submissions have lists of figures/tables at the end of the paper and that
% figures/tables without any captions are placed on a page by themselves at
% the end of the document. If needed, the draftcls IEEEtran class option or
% \CLASSINPUTbaselinestretch interface can be used to increase the line
% spacing as well. Be sure and use the nomarkers option of endfloat to
% prevent endfloat from "marking" where the figures would have been placed
% in the text. The two hack lines of code above are a slight modification of
% that suggested by in the endfloat docs (section 8.4.1) to ensure that
% the full captions always appear in the list of figures/tables - even if
% the user used the short optional argument of \caption[]{}.
% IEEE papers do not typically make use of \caption[]'s optional argument,
% so this should not be an issue. A similar trick can be used to disable
% captions of packages such as subfig.sty that lack options to turn off
% the subcaptions:
% For subfig.sty:
% \let\MYorigsubfloat\subfloat
% \renewcommand{\subfloat}[2][\relax]{\MYorigsubfloat[]{#2}}
% However, the above trick will not work if both optional arguments of
% the \subfloat command are used. Furthermore, there needs to be a
% description of each subfigure *somewhere* and endfloat does not add
% subfigure captions to its list of figures. Thus, the best approach is to
% avoid the use of subfigure captions (many IEEE journals avoid them anyway)
% and instead reference/explain all the subfigures within the main caption.
% The latest version of endfloat.sty and its documentation can obtained at:
% http://www.ctan.org/pkg/endfloat
%
% The IEEEtran \ifCLASSOPTIONcaptionsoff conditional can also be used
% later in the document, say, to conditionally put the References on a 
% page by themselves.




% *** PDF, URL AND HYPERLINK PACKAGES ***
%
%\usepackage{url}
% url.sty was written by Donald Arseneau. It provides better support for
% handling and breaking URLs. url.sty is already installed on most LaTeX
% systems. The latest version and documentation can be obtained at:
% http://www.ctan.org/pkg/url
% Basically, \url{my_url_here}.




% *** Do not adjust lengths that control margins, column widths, etc. ***
% *** Do not use packages that alter fonts (such as pslatex).         ***
% There should be no need to do such things with IEEEtran.cls V1.6 and later.
% (Unless specifically asked to do so by the journal or conference you plan
% to submit to, of course. )


% correct bad hyphenation here
\hyphenation{op-tical net-works semi-conduc-tor}

\usepackage[margin=.75in]{geometry}
\usepackage{courier}
\usepackage{ifthen}
\usepackage{setspace}
\usepackage{listings}
\usepackage[usenames, dvipsnames]{color}
\usepackage{tabularx}
\usepackage[strict]{chngpage}
\usepackage{cite}

\lstset {
	language=C,
	basicstyle=\ttfamily,
	keywordstyle=\color{blue}\ttfamily,
	stringstyle=\color{red}\ttfamily,
	commentstyle=\color{OliveGreen}\ttfamily,
	morecomment=[l][\color{magenta}]{\#}
	showstringspaces=false,
	showspaces=false,
	frame=single,
	captionpos=b
}

\newcommand{\commandline}[2][\empty]
{
\begin{quote}
\texttt{#2}
\ifthenelse{\equal{#1}{\empty}}{}{\begin{quote}#1\end{quote}}
\end{quote}
}

\begin{document}
\singlespace




%
% paper title
% Titles are generally capitalized except for words such as a, an, and, as,
% at, but, by, for, in, nor, of, on, or, the, to and up, which are usually
% not capitalized unless they are the first or last word of the title.
% Linebreaks \\ can be used within to get better formatting as desired.
% Do not put math or special symbols in the title.
\title{\vspace{2in}Project 1: Getting Acquainted}
%
%
% author names and IEEE memberships
% note positions of commas and nonbreaking spaces ( ~ ) LaTeX will not break
% a structure at a ~ so this keeps an author's name from being broken across
% two lines.
% use \thanks{} to gain access to the first footnote area
% a separate \thanks must be used for each paragraph as LaTeX2e's \thanks
% was not built to handle multiple paragraphs
%

\author{Albert Morgan}

% note the % following the last \IEEEmembership and also \thanks - 
% these prevent an unwanted space from occurring between the last author name
% and the end of the author line. i.e., if you had this:
% 
% \author{....lastname \thanks{...} \thanks{...} }
%                     ^------------^------------^----Do not want these spaces!
%
% a space would be appended to the last name and could cause every name on that
% line to be shifted left slightly. This is one of those "LaTeX things". For
% instance, "\textbf{A} \textbf{B}" will typeset as "A B" not "AB". To get
% "AB" then you have to do: "\textbf{A}\textbf{B}"
% \thanks is no different in this regard, so shield the last } of each \thanks
% that ends a line with a % and do not let a space in before the next \thanks.
% Spaces after \IEEEmembership other than the last one are OK (and needed) as
% you are supposed to have spaces between the names. For what it is worth,
% this is a minor point as most people would not even notice if the said evil
% space somehow managed to creep in.



% The paper headers
\markboth{Operating Systems II, Oregon State University}%
{Assignment 1}
% The only time the second header will appear is for the odd numbered pages
% after the title page when using the twoside option.
% 
% *** Note that you probably will NOT want to include the author's ***
% *** name in the headers of peer review papers.                   ***
% You can use \ifCLASSOPTIONpeerreview for conditional compilation here if
% you desire.




% If you want to put a publisher's ID mark on the page you can do it like
% this:
%\IEEEpubid{0000--0000/00\$00.00~\copyright~2015 IEEE}
% Remember, if you use this you must call \IEEEpubidadjcol in the second
% column for its text to clear the IEEEpubid mark.



% use for special paper notices
%\IEEEspecialpapernotice{(Invited Paper)}




% make the title area
% Disable the IEEE standard title and make a custom title page
% As a general rule, do not put math, special symbols or citations
% in the abstract or keywords.
% \vspace{2in}
% \maketitle
\pagestyle{empty}
\vspace*{2in}
\begin{center}
\huge
I/O\\
\normalsize
Operating Systems II\\
Spring 2016\\
Albert Morgan
\end{center}
\vspace{.5in}
\begin{center}
\textbf{Abstract}
\end{center}
\begin{adjustwidth}{2in}{2in}
In his report, \textit{I/O}, OSU computer science major
Albert Morgan compares how the Linux, BSD, and Windows operating
systems handle I/O, with special attention paid to block I/O.
Morgan's report uses research from four different technical
textbooks and two offical online sources to create summaries of
each operating system and compare their functionality side-by-side.
Morgan uses these comparisons to make the point that Linux and BSD
are very similiar, but the I/O functionality of Windows is
extremely different, both in terms of internal data structures and
prioritization algorithms. Morgan's report is written in an
academic manner for an audience already familiar with the internal
components of operating systems.
\end{adjustwidth}
\newpage
\pagestyle{headings}
% Note that keywords are not normally used for peerreview papers.
%\begin{IEEEkeywords}
%IEEE, IEEEtran, journal, \LaTeX, paper, template.
%\end{IEEEkeywords}
%
% For peer review papers, you can put extra information on the cover
% page as needed:
% \ifCLASSOPTIONpeerreview
% \begin{center} \bfseries EDICS Category: 3-BBND \end{center}
% \fi
%
% For peerreview papers, this IEEEtran command inserts a page break and
% creates the second title. It will be ignored for other modes.
%\IEEEpeerreviewmaketitle
%
\section{Introduction}
This report will comapre the way that Linux, BSD, and
Windows handle I/O. I/O is the way that the operating system
communicates with external devices such as hard drives and keyboards,
and efficiently handling of I/O can be a significant performance
consideration.

There are two types of I/O devices: block devices and character devices.
Character devices are a simple stream of data. Reading from a character
device is a relatively straightforward process: simply get the next character.
Block devices, on the other
hand, are much more complicated. Different threads can be attempting to read
and write from and to different sections of the block device at the same time.
Additionally, block devices such as disk drives are very slow due to the
mechanical components, making efficient ordering of I/O requests an important
consideration. Because block I/O is such a more important and
complicated topic, that will be the focus of this report.
%
\subsection{Linux}
Block I/O in Linux is handled by three structures, \textit{requests},
\textit{bios}, and \textit{bio vectors}. When a system call is made that
needs to read or write data on a block device,
\texttt{request} is created. A request
represents a series of contiguous sectors
on the device. The memory that the sectors are going to be loaded in to,
however, can be anywhere. If two requests are adjacent on the block device,
they can be merged into a single, more efficient request. However, read
and write requests can never be merged together.

Each \texttt{request} contains a linked list of \texttt{bio} (block I/O) structs, which
are the workhorses of the \texttt{request}. A \texttt{bio} struct is a single
operation on the block device. Listing~\ref{biostruct} shows the
\texttt{bio} struct.

There is one more layer that must be considered before we are finished. Each
\texttt{bio} struct contains a number of {bi\_vec} structs, which map
the blocks read off the device direcly to memory pages. This enabled
\textit{scatter/gather}, in which contiguous segments on a disk do not need
to be mapped to contiguous blocks of memory~\cite{ldd3}.

\lstinputlisting[language=c, firstline=42, lastline=105,
caption={The Linux kernel's \texttt{bio} struct,
found in \texttt{block\_types.h}},
label={biostruct}]{blk_types.h}

Once the request is made, the kernel must 
decide the order in which requests are processed. This is important because
most block devices, such as disk drives, do not have an instantaneous
seek time; it takes a small but noticible amount of time for disk head
to traverse the span of the disk. By ordering the requests in an intelligent
manner, overall efficiency can be improved. The algorithms that determine
the order that requests are serviced are known as \textit{schedulers},
which are also sometimes called \textit{elevators}.
The Linux kernel comes with three schedulers.

The \textit{noop} scheduler is the simplest. Designed for devices with
no seek time such as solid-state drives, the noop scheduler is simply
a FIFO queue, with each request being serviced in the order it was
received.

Depending on the algorithm used to schedule requests, some requests might
take a very long time to be serviced, or even not serviced at all. This
condition is known as \textit{starvation}. The \textit{deadline} scheduler
was written to fix starvation issues in previous scheduler algorithms. The
deadline scheduler keeps three queues: a read, write, and sorted queue.
When a request comes in, it is given an expiration time,
by default 500ms for read requests and 5 seconds for write requests. Read
requests get a much smaller expiration time because applications typically
block when they are reading, and fulfilling these requests faster will
result in a more responsive system.
Requests are inserted two queues: both the sorted queue and the read or write
queue, whichever is appropriate. Under normal operation, requests are simply
serviced from the sorted queue. However, if the request at the front of the
read or write queue expires, the scheduler will start to service that
queue instead. By keeping track of the expiration times, the deadline
scheduler can prevent starvation and make sure all requests are serviced in
a reasonable amount of time. The 
\textit{anticipatory} scheduler tried to improve
on this algorithm by having the disk head linger after making a read in
case another read was issued shortly thereafter. However, the anticipatory
scheduler was removed from the kernel in version  favor of the
CFQ scheduler in kernel version 2.6.33.

The \textit{CFQ}, or Complete Fair Queueing I/O scheduler is similiar to the
Completely Fair Scheduler that the Linux kernel uses for process management
in that is attempts to divide resources as 
evenly as possible between processes. Its operation is very simple: a
separate queue is maintained for each process on the system, and when
a new request comes in, it is added to the appropriate queue. The queues
are then serviced round robin. The CFQ scheduler works well for
multimedia applications that need to access the hard drive at a relatively
constant rate, and is currently the default scheduler for the Linux
operating system\cite{linux-kernel-dev}.

\subsection{BSD}
Block I/O in BSD works very similarlly to Linux. Instead of using
a \texttt{bio} struct to handle operations, BSD uses a struct called
\texttt{uio}\cite{des-imp-bsd-os}. Listing~\ref{uiostruct} shows
the \texttt{uio} struct. 
Important note: BSD also has a struct called \texttt{bio},
but it is used as part of the GEOM layer which enables arbitrary I/O
transformations and is used for things like RAID or encrypted disks.
This struct is not the same as the Linux kernel's \texttt{bio}
struct~\cite{geom}.
Similiar to Linux, the \texttt{uio} struct uses a list of vectors
to map memory pages to sectors on the device.

\lstinputlisting[language=c, firstline=32, lastline=40,
caption={The BSD kernel's \texttt{uio} struct, found in \texttt{uio.h}},
label={uiostruct}]{uio.h}

One key difference between BSD and Linux block I/O is how requests are
sorted. Linux provides three different schedulers, two of which are
relatively complicated. In contrast, the BSD kernel provides one generic
scheduler: \texttt{disksort()}. The \texttt{disksort()} scheduler implements
the \textit{C-LOOK} algorithm. The scheduler maintains two queues, an active
queue and a next-pass queue. Requests are serviced in sector order.
Requests that come after the current position
of the disk head are sorted into the active list, and other requests
are sorted into the next-pass queue. When the active queue is empty,
the next-pass queue becomes the active queue, and a new next-pass queue
is created\cite{des-imp-bsd-os}.

\subsection{Windows}
Windows I/O is as different from Linux as BSD is similar.
One notable philosophical distinction between Windows and unix-based
systems is that in Windows, an I/O operation allows communication with
a device, and files are treated as special kinds of devices.
This is the opposite of the unix philosophy, which prefers to think
of everything (even devices) as files\cite{win-int-two}.

The core of the Windows I/O system is the I/O Request Packet, or
\texttt{IRP}. The \texttt{IRP} struct is partially opaque, and
Listing ~\ref{irpstruct} shows the members of the
\texttt{IRP} struct that drivers can use~\cite{msdn}.

\lstinputlisting[language=c, caption={The Windows \texttt{IRP} struct\cite{msdn}}, label={irpstruct}]{irp.h}

Like Linux and BSD, Windows supports scatter/gather I/O. The
\texttt{MdlAddress} member of the \texttt{IRP} struct is similiar
to the vectors that Linux's \texttt{bio} struct and BSD's
\texttt{uio} struct uses. An MDL is a \textit{memory description list},
which represents the memory buffers that the I/O operation should use.
A scatter/gather operation is started and stopped with the functions
\texttt{GetScatterGatherList()} and \texttt{PutScatterGatherList()}, which
takes the \texttt{IRP}'s MDL as an argument.

Windows block I/O scheduling is extremely different from either 
Linux or BSD. Much like its process scheduler, Windows gives
all of the resources to the highest priority process, and uses a system
of boosts to try to keep operations from starving.
Two types of prioritization: individual priority and bandwidth
reservations, which will be explained below.

Individual priority operations can be divided two different
\textit{strategies}: \textit{hierarchy} or \textit{idle}.
The hierarchy divides I/O operations into five
categories: critial, high, normal, low, and very low. In the current
version of Windows, only critical, normal, and very low are used, with high
and low included for possible use in future versions. The rules
are simple: each priority class of I/O can only be processed if there
is nothing of higher priority to process, e.g. a normal operation will
not be processed if there are any critical operations in flight. Operations
using the idle strategy will be processed after all hierarchical operations.

Windows uses a few different methods to prevent starvation of operations.
The idle queue is guaranteed to process an operation every half a second or
so. To prevent \textit{priority inversion}, a situation that arises when
a high I/O priority thread is waiting on a low I/O priority thread,
the low priority thread will be given a temporary boost to get its
operation done faster. Boosts will also be given when a higher-priority
process tries to access a lower-priority file, or when a low-priority
process accesses the paging file. This combination of hierarchies and
boosts covers the individual priority system.

The next way that Windows handles I/O prioritization is
\textit{bandwidth reservation}. This simply means that the kenrel
uses time multiplexing to give certain applications highest priority
during dedicated timeslices. For example, a media application might
use bandwidth reservation to ensure that its I/O operations are being
services in a timely manner and it can
keep its buffer full~\cite{win-int-two}.

\section{Conclusion}
Linux and BSD are almost identical in the way they handle I/O. The
specific names and data structures differ slightly, but the basic
idea of requests, I/O operations, and vectors remains the same. One
difference is that BSD uses the C-LOOK scheduler, while Linux uses
the much more complicated CFQ scheduler by default.

Windows, on the other hand, is completely different. Although it
supports the same types of things that Linux and BSD do, like
scatter/gather I/O, the internal data structures and philosophies
behind the algorithms are completely different.

The Windows kernel also applies the unique idea of boosting to
its I/O scheduler. Similiar to the way it schedules processes,
Windows does not try to maximize fairness or minimize overall
seek times the way that Linux and BSD do. The Windows I/O scheduler
services all requests from high priority to low priority, and applies
a relatively complicated system of boosts to ensure that no
problems arise. This boosting system is completely unlike the I/O
systems of Linux or BSD.
%
% An example of a floating figure using the graphicx package.
% Note that \label must occur AFTER (or within) \caption.
% For figures, \caption should occur after the \includegraphics.
% Note that IEEEtran v1.7 and later has special internal code that
% is designed to preserve the operation of \label within \caption
% even when the captionsoff option is in effect. However, because
% of issues like this, it may be the safest practice to put all your
% \label just after \caption rather than within \caption{}.
%
% Reminder: the "draftcls" or "draftclsnofoot", not "draft", class
% option should be used if it is desired that the figures are to be
% displayed while in draft mode.
%
%\begin{figure}[!t]
%\centering
%\includegraphics[width=2.5in]{myfigure}
% where an .eps filename suffix will be assumed under latex, 
% and a .pdf suffix will be assumed for pdflatex; or what has been declared
% via \DeclareGraphicsExtensions.
%\caption{Simulation results for the network.}
%\label{fig_sim}
%\end{figure}

% Note that the IEEE typically puts floats only at the top, even when this
% results in a large percentage of a column being occupied by floats.


% An example of a double column floating figure using two subfigures.
% (The subfig.sty package must be loaded for this to work.)
% The subfigure \label commands are set within each subfloat command,
% and the \label for the overall figure must come after \caption.
% \hfil is used as a separator to get equal spacing.
% Watch out that the combined width of all the subfigures on a 
% line do not exceed the text width or a line break will occur.
%
%\begin{figure*}[!t]
%\centering
%\subfloat[Case I]{\includegraphics[width=2.5in]{box}%
%\label{fig_first_case}}
%\hfil
%\subfloat[Case II]{\includegraphics[width=2.5in]{box}%
%\label{fig_second_case}}
%\caption{Simulation results for the network.}
%\label{fig_sim}
%\end{figure*}
%
% Note that often IEEE papers with subfigures do not employ subfigure
% captions (using the optional argument to \subfloat[]), but instead will
% reference/describe all of them (a), (b), etc., within the main caption.
% Be aware that for subfig.sty to generate the (a), (b), etc., subfigure
% labels, the optional argument to \subfloat must be present. If a
% subcaption is not desired, just leave its contents blank,
% e.g., \subfloat[].


% An example of a floating table. Note that, for IEEE style tables, the
% \caption command should come BEFORE the table and, given that table
% captions serve much like titles, are usually capitalized except for words
% such as a, an, and, as, at, but, by, for, in, nor, of, on, or, the, to
% and up, which are usually not capitalized unless they are the first or
% last word of the caption. Table text will default to \footnotesize as
% the IEEE normally uses this smaller font for tables.
% The \label must come after \caption as always.
%
%\begin{table}[!t]
%% increase table row spacing, adjust to taste
%\renewcommand{\arraystretch}{1.3}
% if using array.sty, it might be a good idea to tweak the value of
% \extrarowheight as needed to properly center the text within the cells
%\caption{An Example of a Table}
%\label{table_example}
%\centering
%% Some packages, such as MDW tools, offer better commands for making tables
%% than the plain LaTeX2e tabular which is used here.
%\begin{tabular}{|c||c|}
%\hline
%One & Two\\
%\hline
%Three & Four\\
%\hline
%\end{tabular}
%\end{table}


% Note that the IEEE does not put floats in the very first column
% - or typically anywhere on the first page for that matter. Also,
% in-text middle ("here") positioning is typically not used, but it
% is allowed and encouraged for Computer Society conferences (but
% not Computer Society journals). Most IEEE journals/conferences use
% top floats exclusively. 
% Note that, LaTeX2e, unlike IEEE journals/conferences, places
% footnotes above bottom floats. This can be corrected via the
% \fnbelowfloat command of the stfloats package.



% if have a single appendix:
%\appendix[Proof of the Zonklar Equations]
% or
%\appendix  % for no appendix heading
% do not use \section anymore after \appendix, only \section*
% is possibly needed

% use appendices with more than one appendix
% then use \section to start each appendix
% you must declare a \section before using any
% \subsection or using \label (\appendices by itself
% starts a section numbered zero.)

% Can use something like this to put references on a page
% by themselves when using endfloat and the captionsoff option.



% trigger a \newpage just before the given reference
% number - used to balance the columns on the last page
% adjust value as needed - may need to be readjusted if
% the document is modified later
%\IEEEtriggeratref{8}
% The "triggered" command can be changed if desired:
%\IEEEtriggercmd{\enlargethispage{-5in}}

% references section

% can use a bibliography generated by BibTeX as a .bbl file
% BibTeX documentation can be easily obtained at:
% http://mirror.ctan.org/biblio/bibtex/contrib/doc/
% The IEEEtran BibTeX style support page is at:
% http://www.michaelshell.org/tex/ieeetran/bibtex/
%\bibliographystyle{IEEEtran}
% argument is your BibTeX string definitions and bibliography database(s)
%\bibliography{IEEEabrv,../bib/paper}
%
% <OR> manually copy in the resultant .bbl file
% set second argument of \begin to the number of references
% (used to reserve space for the reference number labels box)

\bibliography{mybib}
\bibliographystyle{IEEEtran}

% biography section
% 
% If you have an EPS/PDF photo (graphicx package needed) extra braces are
% needed around the contents of the optional argument to biography to prevent
% the LaTeX parser from getting confused when it sees the complicated
% \includegraphics command within an optional argument. (You could create
% your own custom macro containing the \includegraphics command to make things
% simpler here.)
%\begin{IEEEbiography}[{\includegraphics[width=1in,height=1.25in,clip,keepaspectratio]{mshell}}]{Michael Shell}
% or if you just want to reserve a space for a photo:


% You can push biographies down or up by placing
% a \vfill before or after them. The appropriate
% use of \vfill depends on what kind of text is
% on the last page and whether or not the columns
% are being equalized.

%\vfill

% Can be used to pull up biographies so that the bottom of the last one
% is flush with the other column.
%\enlargethispage{-5in}



% that's all folks
\end{document}


